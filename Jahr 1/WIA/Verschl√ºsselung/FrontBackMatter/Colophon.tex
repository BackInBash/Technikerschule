% Colophon (a brief description of publication or production notes relevant to the edition)

\pagestyle{empty}

\hfill

\vfill

\pdfbookmark[0]{Colophon}{colophon}

\section*{Colophon}

This document was typeset using the typographical look-and-feel \texttt{classicthesis} developed by Andr\'e Miede. The style was inspired by Robert Bringhurst's seminal book on typography ``\emph{The Elements of Typographic Style}''. \texttt{classicthesis} is available for both \LaTeX\ and \mLyX: 

\begin{center}
\url{https://bitbucket.org/amiede/classicthesis/}
\end{center}

\noindent Dieses Paper ist unter der MIT Lizenz veröffentlicht der Quellcode ist auf GitHub abrufbar.

\begin{center}
\url{https://github.com/BackInBash/Technikerschule/tree/master/Jahr%201/WIA/Verschl%C3%BCsselung}
\end{center}
 
\bigskip

\noindent\finalVersionString